\section{Conclusion}

\subsection{Summary}
In this work, we design the critical component for a railroad trespassing detection system. Although initially envisioned for railroad security, the proposed approach has potential applications in video surveillance domains characterized by a sparsity in activity. The contributions of this thesis include a flexible pipeline that can trade off speed and accuracy. The system by design consists of two stages where the first stage is responsible for efficiently removing the background frames from the activity frames. The second stage is responsible for differentiating between human trespassing activity and any other unknown activity. Our proposed pipeline is composed of off-the-shelf components. Other algorithms relevant to stage 1 and stage 2 could equally be plugged in. We demonstrate the effectiveness of our approach on a public domain surveillance dataset. 

\subsection{Future work}
There are many interesting directions in which this work can be further advanced. One key direction is to build a trespassing prediction system that uses the output of this detection system to predict trespassing events in near future. Another direction is towards improving the accuracy of detection system. We note that the current performance is limited by the performance of stage 2. Currently stage 2 doesn't use any temporal information i.e. each frame is treated independently and is not conditioned on the previous frames (history). We believe that utilizing the temporal information can significantly improve the performance specially for challenging cases of occlusion and background (discussed in section \ref{sec:challenges}). 

Furthermore, an approach towards reducing the processing time could be to reduce the number of proposals (Regions of Interest) generated by Faster-RCNN. Since, we are interested in detecting human trespassers only, we can use the foreground mask to produce RoI (Region of Interest) proposals. Suppressing the color information corresponding to the background area should significantly reduce the number of proposals. Also, the whole process can be simplified by considering the foreground areas as proposals and classifying them directly with a classification network as opposed to using Faster-RCNN. 

\newpage