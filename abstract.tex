
\begin{center} 
\textbf{Abstract}
\end{center}
While railroad trespassing is a dangerous activity with significant security and safety risks, regular patrolling of potential trespassing sites is infeasible due to exceedingly high resource demands and personnel costs. There is thus a need to design an automated trespass detection and early warning prediction tool leveraging state-of-the-art machine learning techniques.  Leveraging video surveillance through security cameras, this thesis designs a novel approach called ARTS (Automated Railway Trespassing detection System) that tackles the problem of detecting trespassing activity.  In particular, we adopt a CNN-based deep learning architecture (Faster-RCNN) as the core component of our solution. However, these deep learning-based methods, while effective, are known to be computationally expensive and time consuming, especially when applied to a large amount of surveillance data. Given the sparsity of railroad trespassing activity, we design a dual-stage deep learning architecture composed of an inexpensive prefiltering stage for activity detection followed by a high fidelity trespass detection stage for robust classification.  The former is responsible for filtering out frames that show little to no activity, this way reducing the amount of data to be processed by the later more compute-intensive stage which adops state-of-the-art Faster-RCNN to ensure effective classification of trespassing activity.  The resulting dual-stage architecture ARTS represents a flexible solution capable of trading-off performance and computational time. We demonstrate the efficacy of our approach on a public domain surveillance dataset.
\newpage